\documentclass[a4paper, titlepage, french]{report}

\usepackage{amsmath,amsthm,amsfonts,delarray,fourier,manfnt}
\usepackage[utf8]{inputenc} % accents
\usepackage[T1]{fontenc}      % caractères français
\usepackage[french]{babel}  % langue
\usepackage{geometry,graphicx}         % marges % images
\usepackage{lmodern,eso-pic,fullpage}  
\usepackage{verbatim}         % texte préformaté
\usepackage{stackengine,scalerel}
\usepackage{authblk}
\usepackage[linesnumbered,algosection]{algorithm2e} %Pour le PDL
\usepackage{listings, xcolor}
\usepackage{moreverb,wasysym}
\usepackage{pgf, tikz} %pour dessiner
\usepackage{multicol} %pour avoir multiples colonnes localement
\usepackage{titlesec} %pour enlever le mot "chapitre"
\usepackage{lipsum} %ajouter du texte dans les dessins tikz
\usepackage{fancyvrb} %verbatim mais en mieux on l'utilise avec \begin{BVerbatim}, et cette fois on peut centrer
\usepackage{moreverb} %verbatim en box ebvironnement boxedverbatim
\usepackage{mdframed}

%\SetKwIF{Si}{SinonSi}{Sinon}{si}{alors}{sinon si}{alors}{finSi}



\usepackage[colorlinks=true, allcolors=black]{hyperref} %pour utiliser des liens internes, càd en réferencant une partie en peut mettre un lien vers celle-ci et ça avec \hyperref[nom du label]{le texte où y'aura le lien} sans oublier de mettre le \label{nom du label} sur la partie que vous voulez réferencer
\usepackage[bottom]{footmisc} %remettre le compteur de footnote à 0 par section.
\counterwithin*{footnote}{section} %numerote les footnote par section

                          % Le PDL
%les fonctions et leurs types

%pour les entiers
\SetKwProg{FnInt}{ \textcolor{blush}{entier} \textcolor{babyblueeyes}{fonction} }{}{\textcolor{babyblueeyes}{fin Fonction}}

%pour les caractères
\SetKwProg{FnChar}{ \textcolor{blush}{car} \textcolor{babyblueeyes}{fonction} }{}{\textcolor{babyblueeyes}{fin Fonction}}

%pour les booléens
\SetKwProg{FnBool}{ \textcolor{blush}{booléen} \textcolor{babyblueeyes}{fonction} }{ }{\textcolor{babyblueeyes}{fin Fonction}}

%pour les tableaux de dimension 1
\SetKwProg{FnTab}{\textcolor{blush}{tab[]} \textcolor{babyblueeyes}{fonction} }{}{\textcolor{babyblueeyes}{fin Fonction}}

%pour les tableaux de caractères
\SetKwProg{FnCharTab}{\textcolor{blush}{car[ ]} \textcolor{babyblueeyes}{fonction}  }{}{\textcolor{babyblueeyes}{fin Fonction}}

%pour les positions
\SetKwProg{FnPos}{ \textcolor{blush}{Position} \textcolor{babyblueeyes}{fonction} }{}{\textcolor{babyblueeyes}{fin Fonction}}

%pour les matrices de dimension 2
\SetKwProg{FnMat}{\textcolor{blush}{tab[][]} \textcolor{babyblueeyes}{fonction}}{}{\textcolor{babyblueeyes}{fin Fonction}}

%pour les réels
\SetKwProg{FnReel}{\textcolor{blush}{réel} \textcolor{babyblueeyes}{fonction} }{}{\textcolor{babyblueeyes}{fin Fonction}}

%Actions
\SetKwProg{Act}{\textcolor{chromeyellow}{action}}{}{\textcolor{chromeyellow}{fin Action}}

% \newmdenv[linewidth=0, leftmargin= -14,rightmargin= 14, backgroundcolor=red!10]{sourcecode}

%pour ajouter des commentaires dans l'algo 
%\tcp{. . .}

%couleur key words
\definecolor{babyblueeyes}{rgb}{0.63, 0.79, 0.95}%joli bleu
\definecolor{blush}{rgb}{0.87, 0.36, 0.51} %joli rose
\definecolor{chromeyellow}{rgb}{1.0, 0.65, 0.0} %joli jaune\orange pour julie 
\definecolor{camouflagegreen}{rgb}{0.47, 0.53, 0.42} %joli vert
\definecolor{cardinal}{rgb}{0.77, 0.12, 0.23} %joli rouge


\titleformat{\chapter}[display]
  {\normalfont\bfseries}{}{0pt}{\Huge}

\newcommand{\HRule}{\rule{\linewidth}{0.5mm}}
\newcommand{\blap}[1]{\vbox to 0pt{#1\vss}}
\newcommand\AtUpperLeftCorner[3]{%
  \put(\LenToUnit{#1},\LenToUnit{\dimexpr\paperheight-#2}){\blap{#3}}%
}
\newcommand\AtTopCenterPage[2]{%
  \put(\LenToUnit{.5\paperwidth},\LenToUnit{\dimexpr\paperheight-#1}){\blap{\hbox to 0pt{\hss#2\hss}}}%
}

\newcommand\AtUpperRightCorner[3]{%
  \put(\LenToUnit{\dimexpr\paperwidth-#1},\LenToUnit{\dimexpr\paperheight-#2}){\blap{\llap{#3}}}%
}

\lstset{
  basicstyle=\ttfamily,
  mathescape=true
}

\definecolor{purple}{rgb}{0.57, 0.36, 0.51}

%algo
\RestyleAlgo{boxed}
 \SetAlgoSkip{1cm}
 
 



\title{\LARGE{Projet d'APOO} \\[0,5cm] \LARGE{Analyse du jeu de blocage avec polyominos}}
\author{   \\ Antoine \textsc{CUINET}  \\ Gaspar \textsc{QUENTIN}   }
\makeatletter


\begin{document}


\selectlanguage{french}
\begin{titlepage}
 \enlargethispage{3cm}
 
     \AddToShipoutPicture{
        \AtUpperLeftCorner{1.5cm}{1cm}{\includegraphics[width=6cm]{ufclogo.jpg}} 
    }

 \begin{center}	

        \vspace*{8cm}
 
        \textsc{\@title} \\
	\vspace*{0,5cm}
        \HRule
 	 \vspace*{0,5cm}
	\large{\@author} 
    \end{center}

  \vspace*{9cm}

   \end{titlepage}

\ClearShipoutPicture

%si on veut rajouter une page vide
%\newpage
%\thispagestyle{empty}
%\mbox{}
%\newpage

\tableofcontents
% \listofalgorithms

\newpage



















\chapter{Introduction}

Le projet est un projet de fin d'année de première année de licence informatique à l'université de Franche-Compté.

\bigskip

Ce rapport décrit et explique la conception de ce projet, qui est le jeu de blocage avec polyominos.

\bigskip

Le jeu est codé en Java et le jouer joue contre un ordinateur depuis le terminal.

\bigskip

Le but du jeu est relativement simple, chaque joueur dispose d'un ensemble de pièces, qui sont des polyominos, et qu'il place
tour à tour sur un plateau, qui est sous forme de grille rectangulaire. 

Il s'agit d'un jeu de blocage: le perdant est le premier
joueur qui ne peut plus placer de pièce.

Le jeu se déroule en mode
humain contre ordinateur: le joueur humain interagit via des affichages et des saisies sur la console.

\bigskip

Ce jeu à été coder par CUINET Antoine et QUENTIN Gaspard.



\newpage

\chapter{Les fonctionnalités}

Les fonctionnalités du jeu sont multiples.

Le joueur peu

\color{red}
// ici à faire
\color{black}

\chapter{Les classe}

Le jeu dispose de 10 classes.
\bigskip

\color{red}
// faire les classes, bonne chance à celui qui le fera
\color{black}

\section{La classe Main}
\bigskip

\section{La classe Computeur}
\bigskip

\section{La classe Piece}
\bigskip

\section{La classe Grid}
\bigskip

\section{La classe Position}
\bigskip

\section{La classe Case}
\bigskip

\section{La classe Matrix}
\bigskip

\section{La classe Domino}
\bigskip

\section{La classe Triomino}
\bigskip

\section{La classe Tetromino}
\bigskip

\chapter{Initialisation de jeu}

Au début du jeu, un affichage permettant de saisir de pseudo du jeueur est mis à l'écran.

Une fois cette saisie faite, une explication du jeu, de l'ordinateur (qui se nome Yumi), ainsi que des pièces disponible est présenté.



\section{Dimensions du jeu}

Le plateau de jeu est un rectangle de 12 cases par 10 cases, qui sont numérotés en lettres (de A à L) en abscisse et en chiffre (de 0 à 9) en ordonné.

\section{Distribution des pièces}

Au début du jeu, le joueur comme l'ordinateur ont à disposition un même nombre de pièce.
\bigskip

Il est distribué 3 Dominos, 6 Triomino posable en 2 positions avec 3 pièces disponibles par position, et 9 Tétrominos posable en 7 positions avec 1 pièce disponible par position sauf pour les tétrominos de formes I et T qui sont aux nombre de 2.

Ainsi, chaque joueur un nobre de 60 cases sur un plateau faisant en tout 120 cases.

\chapter{Description des conditions}

Plusieur types de conditions sont utilisé afin de faire fonctionner au mieux le jeu.

\section{Placement des pièces}

\color{red}
// ici à faire
\color{black}


\section{La partie est finie}

\color{red}
// ici à faire
\color{black}


\chapter{Jeu de l'ordinateur}

Par manque de temps pour réaliser ce projet, notre ordinateur choisi ses pièce et les places aléatoirement.
\chapter{Ajouts/libertés prises}

Afin de rajouter de la créativité, l'affichage de la grille est réalisé de façon originale.
\bigskip

Le placement des pièces dans la grilles se fait en couleur, de façon à distinguer facilement et faire un affichage coloré (rouge pour l'ordinateur et bleu pour le joueur).
\bigskip

Le nom du jeueur ainsi que celui de l'ordinateur sont eux aussi en couleur (de même, rouge pour l'ordinateur et bleu pour le joueur).
\bigskip

Enfin, l'affichage des pièces en vert, qui est guidé tout au long du processus de choix d'une pièce, permet au joueur de facilement se rendre compte des pièces possible de poser.


\chapter{Conclusion}

Pour conclure ce rapport, nous avons produit un jeu amusant, autant dans la conception que dans son utilisation.
\bigskip

Certain problèmes ont été rencontrer comme celui de faire les tétrominons qui à été très long (autant dans la création des pièce que dans les affichages).
De plus, celui de distinguer les formes possibles de chaques pièces, en plus des différentes pièces et de leurs orientations.
\bigskip

Ces problèmes ont bien évidement trouver leurs solutions et nous ont permis de nous surpasser dans la création de ce jeu.
\bigskip

Concevoir ce jeu nous à également permis de mettre en pratique les cours appris tout au long de l'année et spécifiquement la programmation orienté objet.

\end{document}